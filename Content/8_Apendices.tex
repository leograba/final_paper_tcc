\chapter{C�digos-fonte da interface de usu�rio}
\label{Ap�ndice A}

C�digos em elabora��o.



% % % % % % % % % % % % % % % % % % % % % % % % % % % % % % % % % % % % % % % % % % % % % % % % % % %
\chapter{Arquivos de configura��o do sistema}
\label{Ap�ndice B}

\section{Arquivo de configura��o de rede}

\lstset{language=bash}
\begin{lstlisting}
# This file describes the network interfaces available
# on your system and how to activate them.
# For more information, see interfaces(5).

# The loopback network interface
auto lo
iface lo inet loopback

# The primary network interface
#auto eth0
#iface eth0 inet dhcp
# Example to keep MAC address between reboots
#hwaddress ether DE:AD:BE:EF:CA:FE

# The secondary network interface
#auto eth1
#iface eth1 inet dhcp

# WiFi Example
auto wlan0
allow-hotplug wlan0
iface wlan0 inet static
#allow-hotplug wlan0
#auto lo
#iface lo inet loopback
#auto eth0
#iface eth0 inet static
address 192.168.1.155
netmask 255.255.252.0
network 192.168.1.0
gateway 192.168.1.1
#pre-up ifconfig eth0 hw ether 00:01:02:03:05:14
dns-nameservers  143.107.225.6 143.107.182.2 8.8.8.8
wpa-ssid        "nome_da_rede"
wpa-psk         "senha1"
#iface wlan0 inet dhcp
#    wpa-ssid "outra_rede"
#    wpa-psk  "senha2"

# Ethernet/RNDIS gadget (g_ether)
# ... or on host side, usbnet and random hwaddr
# Note on some boards, usb0 is automaticly 
#setup with an init script
iface usb0 inet static
address 192.168.7.2
netmask 255.255.255.0
network 192.168.7.0
gateway 192.168.7.1



\end{lstlisting}


% % % % % % % % % % % % % % % % % % % % % % % % % % % % % % % % % % % % % % % % % % % % % % % % % % %
\chapter{Relat�rio Parcial do TCC}
\label{Ap�ndice C}


\begin{center}
\textbf {\color{blue}{Cuidados e orienta��es para a composi��o do Relat�rio Parcial do TCC}}
\end{center}


Trata-se de um relat�rio completo, com todas as partes de uma monografia final. 

Atente-se para as partes em {\color{red}{vermelho}}.

\begin{itemize}
\item Resumo
\item Introdu��o
\item Objetivos
\item Justificativas/Relev�ncia
\item Embasamento Te�rico (Fundamenta��o Te�rica-Revis�o Bibliogr�fica)
\item Materiais e m�todos
\item {\color{red}{Resultados Preliminares}}
\item {\color{red}{Sequ�ncia do trabalho (indicando poss�veis corre��es de rota do projeto)}}
\item {\color{red}{Cronograma Final (com corre��es se necess�rio)}}
\item {\color{red}{Conclus�es Preliminares}}
\item Refer�ncias Bibliogr�ficas
\item Ap�ndices
\item Anexos
\end{itemize} 

Sendo bem feito, ir� poupar esfor�o para a reda��o da monografia.


