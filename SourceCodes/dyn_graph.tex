\lstset{language=HTML}
\begin{lstlisting}[frame=single, basicstyle=\linespread{0.85}\ttfamily\tiny, caption=Gr�fico din�mico de temperatura, label=dyn_html]
<!doctype html>
<head>
	<meta charset="utf-8"/>
	<meta http-equiv="cache-control" content="max-age=0" />
	<meta http-equiv="cache-control" content="no-cache, no-store, must-revalidate" />
	<meta http-equiv="expires" content="0" />
	<meta http-equiv="expires" content="Tue, 01 Jan 1980 1:00:00 GMT" />
	<meta http-equiv="pragma" content="no-cache" />

	<link type="text/css" rel="stylesheet" href="http://ajax.googleapis.com/ajax/libs/jqueryui/1.8/themes/base/jquery-ui.css">
	<link type="text/css" rel="stylesheet" href="./rickshaw/src/css/graph.css">
	<link type="text/css" rel="stylesheet" href="./rickshaw/src/css/detail.css">
	<link type="text/css" rel="stylesheet" href="./rickshaw/src/css/legend.css">
	<link type="text/css" rel="stylesheet" href="./rickshaw/examples/css/extensions.css?v=2">


	<script src="./rickshaw/vendor/d3.v3.js"></script>

	<script src="https://ajax.googleapis.com/ajax/libs/jquery/1.6.2/jquery.min.js"></script>
	<script>
		jQuery.noConflict();
	</script>

	<script src="https://ajax.googleapis.com/ajax/libs/jqueryui/1.8.15/jquery-ui.min.js"></script>

	<script src="./rickshaw/src/js/Rickshaw.js"></script>
	<script src="./rickshaw/src/js/Rickshaw.Class.js"></script>
	<script src="./rickshaw/src/js/Rickshaw.Compat.ClassList.js"></script>
	<script src="./rickshaw/src/js/Rickshaw.Graph.js"></script>
	<script src="./rickshaw/src/js/Rickshaw.Graph.Renderer.js"></script>
	<script src="./rickshaw/src/js/Rickshaw.Graph.Renderer.Area.js"></script>
	<script src="./rickshaw/src/js/Rickshaw.Graph.Renderer.Line.js"></script>
	<script src="./rickshaw/src/js/Rickshaw.Graph.Renderer.Bar.js"></script>
	<script src="./rickshaw/src/js/Rickshaw.Graph.Renderer.ScatterPlot.js"></script>
	<script src="./rickshaw/src/js/Rickshaw.Graph.Renderer.Stack.js"></script>
	<script src="./rickshaw/src/js/Rickshaw.Graph.RangeSlider.js"></script>
	<script src="./rickshaw/src/js/Rickshaw.Graph.RangeSlider.Preview.js"></script>
	<script src="./rickshaw/src/js/Rickshaw.Graph.HoverDetail.js"></script>
	<script src="./rickshaw/src/js/Rickshaw.Graph.Annotate.js"></script>
	<script src="./rickshaw/src/js/Rickshaw.Graph.Legend.js"></script>
	<script src="./rickshaw/src/js/Rickshaw.Graph.Axis.Time.js"></script>
	<script src="./rickshaw/src/js/Rickshaw.Graph.Behavior.Series.Toggle.js"></script>
	<script src="./rickshaw/src/js/Rickshaw.Graph.Behavior.Series.Order.js"></script>
	<script src="./rickshaw/src/js/Rickshaw.Graph.Behavior.Series.Highlight.js"></script>
	<script src="./rickshaw/src/js/Rickshaw.Graph.Smoother.js"></script>
	<script src="./rickshaw/src/js/Rickshaw.Fixtures.Time.js"></script>
	<script src="./rickshaw/src/js/Rickshaw.Fixtures.Time.Local.js"></script>
	<script src="./rickshaw/src/js/Rickshaw.Fixtures.Number.js"></script>
	<script src="./rickshaw/src/js/Rickshaw.Fixtures.RandomData.js"></script>
	<script src="./rickshaw/src/js/Rickshaw.Fixtures.Color.js"></script>
	<script src="./rickshaw/src/js/Rickshaw.Color.Palette.js"></script>
	<script src="./rickshaw/src/js/Rickshaw.Graph.Axis.Y.js"></script>

	<script src="./rickshaw/examples/js/extensions.js"></script>
</head>
<body>
<div id="temp_now" style="display:none;">-</div>
<div id="content">

	<form id="side_panel">
		<!--<h1>Temperatura da Producao</h1>-->
		<section><div id="legend"></div></section>
		<section>
			<div id="renderer_form" class="toggler">
				<input type="radio" name="renderer" id="area" value="area" checked>
				<label for="area">area</label>
				<input type="radio" name="renderer" id="line" value="line">
				<label for="line">linha</label>
			</div>
		</section>
		<section>
			<div id="offset_form">
				<label for="stack">
					<input type="radio" name="offset" id="stack" value="zero" checked>
					<span>pilha</span>
				</label>
				<label for="value">
					<input type="radio" name="offset" id="value" value="value">
					<span>valor</span>
				</label>
			</div>
			<div id="interpolation_form">
				<label for="cardinal">
					<input type="radio" name="interpolation" id="cardinal" value="cardinal" checked>
					<span>cardinal</span>
				</label>
				<label for="linear">
					<input type="radio" name="interpolation" id="linear" value="linear">
					<span>linear</span>
				</label>
				<label for="step">
					<input type="radio" name="interpolation" id="step" value="step-after">
					<span>degrau</span>
				</label>
			</div>
		</section>
		<section>
			<h6>Smoothing</h6>
			<div id="smoother"></div>
		</section>
		<section>
			<h6>Mostrar ultimos:</h6>
			<label for="fiveminutes">
				<input type="radio" name="maxpoints" id="fiveminutes" value="300">
				<span>5 min</span>
			</label>
			<label for="thirtyminutes">
                                <input type="radio" name="maxpoints" id="thirtyminutes" value="1800">
                                <span>30 min</span>
                        </label>
                        <label for="onehour">
                                <input type="radio" name="maxpoints" id="onehour" value="3600" checked>
                                <span>hora</span>
                        </label>
		</section>
		<section></section>
	</form>

	<div id="chart_container">
		<div id="chart"></div>
		<div id="timeline"></div>
		<div id="preview"></div>
	</div>

</div>

<script>


//preisa ter inicializado os dados na hora de plotar, senao da erro
var seriesData =[[], []];
var seriesDataBkp = [[], []];
var graph;//o esquema eh declarar essa global, pq vai modificar dentro de funcao
//variaveis para formatar a string da data para o formato data
var fdata = d3.time.format("%a %b %e %X %Y");
var dt = new Date();
var max_data = 3600;//numero maximo de pontos do grafico
//var diff = [0, 0];//diferenca entre os valores atual e anterior de numero de pontos
//var last_size = [0, 0];//tamanho do array anterior
var last_data = [];//ultima data lida

//le o arquivo csv do log e plota o grafico
//aqui dentro inicializa tudo
d3.csv("./datalog/default.csv", function(data) {//le o arquivo .csv
	console.log("First read from .csv");
	data.forEach(function(d){//para cada valor lido, edita os dados
		d.temperatura = +d.temperatura;//essa string vira numero
		d.data = +d.data;//essa vira Epoch time
		for (var i = 0; i < 2; i++){//le para array temporario
			seriesData[i].push({x:d.data, y:d.temperatura});
			seriesDataBkp[i].push({x:d.data, y:d.temperatura});
			last_data[i] = d.data;//ultima data escrita no array
		}//fim do for
	});//fim do forEach
	//limita o numero de dados plotados excluindo das primeiras posicoes
        seriesData[0].splice(0,seriesData[0].length-max_data);
	seriesData[1].splice(0,seriesData[1].length-max_data);
        seriesDataBkp[0].splice(0,seriesData[0].length-max_data);
        seriesDataBkp[1].splice(0,seriesData[1].length-max_data);
	lets_plot();//plota o grafico
	updateit();
	graph.render();//renderiza pela primeira vez
	console.log(graph);
	console.log(seriesData);
});//fim do d3.csv


//loop que le o arquivo csv com a ultima temperatura valida
setInterval(function(){//realiza o loop a cada x milisegundos
	d3.csv("./datalog/instant.csv", function(data) {//le o arquivo .csv
		var current = [];
		var first_time = 0;
		var j = [];
		var con = [];
		var new_data = [];
		var last_temp;
		var last_temp = document.getElementById("temp_now");
		data.forEach(function(d){//para cada valor lido, edita os dados
			d.temperatura = +d.temperatura;//essa string vira numero
			d.data = +d.data;//essa vira Epoch time
			for (var i = 0; i < 2; i++){//le para array temporario
				current[i] = {x:d.data, y:d.temperatura};
				new_data[i] = d.data;//ultima data lida	
			}//fim do for
			last_temp = d.temperatura;//quando acaba o forEach, tem a ultima temp
			console.log(new_data);
		});//fim do forEach, funcao d
		document.getElementById("temp_now").innerHTML = last_temp+"&degC";//atualiza valor da temp
		console.log(temp_now);
		//console.log("ultima:"+last_temp);
		for (var i = 0; i < 2; i++){//escreve do temporario pro fixo
	        	if(new_data[i] > last_data[i]){//se a data lida eh posterior a ultima data do grafico
				last_data[i] = new_data[i];//atualiza a ultima data do grafico

				if(seriesDataBkp[i].length >= 3600){//backup guarda os ultimos 3600 pontos
                                        seriesDataBkp[i].splice(0,seriesDataBkp[i].length-3600);
                                        seriesDataBkp[i].push(current[i]);
                                }
				else{
					seriesDataBkp[i].push(current[i]);
				}
				if(seriesData[i].length >= max_data){// nunca plota mais que o maximo de pontos
					seriesData[i].splice(0,seriesData[i].length-max_data);
                                        seriesData[i].push(current[i]);
				}
				else{
					if(seriesDataBkp[i].length > (seriesData[i].length+1)){//se mudar escala
						if(seriesDataBkp[i].length < max_data){//se tem menos pontos que o maximo
                                                	j = 0;
                                                	con = seriesDataBkp[i].length-seriesData[i].length;
                                        	}
						else{
                                                        j =  seriesDataBkp[i].length-max_data;
							con = seriesDataBkp[i].length-seriesData[i].length;
						}
						for(; j < con; con--){
							seriesData[i].unshift(seriesDataBkp[i][con]);//completa os elementos anteriores
//por exemplo, se a escala estava 300 e vai pra 1800, mas ja tem 500 pontos, adiciona os 200 pontos iniciais que haviam sido excluidos
						}
					}
					else{
                		        	seriesData[i].push(current[i]);
					}
	                	}
	        	}
		}
		//manda pro console pra debugar
		console.log(seriesData);
		graph.update();
	});//fim do d3.csv
//}, 30000);//repete a cada 15 segundos, metade do tempo do log (30s)
}, 600);//repete mais de uma vez por segundo


//lista dos botoes para ajustar numero de pontos do grafico
var ultimos = document.forms['side_panel'].elements['maxpoints'];
//loop pela lista de botoes
for(var i = 0, len = ultimos.length; i < len; i++){
	ultimos[i].onclick = function(){
		max_data = this.value;//muda o valor do numero maximo de pontos
		console.log("max points changed to max_data"+max_data);
	}
}

// instantiate our graph!

function lets_plot(){
var palette = new Rickshaw.Color.Palette( { scheme: 'cool' } );

 	graph = new Rickshaw.Graph( {
	element: document.getElementById("chart"),
	width: 650,
	height: 300,
	min: 'auto',
	renderer: 'area',
	stroke: true,
	preserve: true,
	series: [
		{
			color: palette.color(),
			data: seriesData[0],
			name: 'Temperatura'
		},{
			color: palette.color(),
			data: seriesData[1],
			name: 'Valvula'
		}
	]
} );

//graph.render();
}

function updateit(){
var preview = new Rickshaw.Graph.RangeSlider( {
	graph: graph,
	element: document.getElementById('preview'),
} );

var hoverDetail = new Rickshaw.Graph.HoverDetail( {
	graph: graph,
	xFormatter: function(x) {
		return new Date(x * 1000).toString();
	}
} );

var annotator = new Rickshaw.Graph.Annotate( {
	graph: graph,
	element: document.getElementById('timeline')
} );

var legend = new Rickshaw.Graph.Legend( {
	graph: graph,
	element: document.getElementById('legend')

} );

var shelving = new Rickshaw.Graph.Behavior.Series.Toggle( {
	graph: graph,
	legend: legend
} );

var order = new Rickshaw.Graph.Behavior.Series.Order( {
	graph: graph,
	legend: legend
} );

var highlighter = new Rickshaw.Graph.Behavior.Series.Highlight( {
	graph: graph,
	legend: legend
} );

var smoother = new Rickshaw.Graph.Smoother( {
	graph: graph,
	element: document.querySelector('#smoother')
} );

var ticksTreatment = 'glow';

var xAxis = new Rickshaw.Graph.Axis.Time( {
	graph: graph,
	ticksTreatment: ticksTreatment,
	timeFixture: new Rickshaw.Fixtures.Time.Local()
	//timeFixture: fdata
} );

xAxis.render();

var yAxis = new Rickshaw.Graph.Axis.Y( {
	graph: graph,
	//pixelsPerTick: graph.height/10,
	tickSize: 10,
	tickFormat: Rickshaw.Fixtures.Number.formatKMBT,
	ticksTreatment: ticksTreatment,
	
} );


yAxis.render();


var controls = new RenderControls( {
	element: document.querySelector('form'),
	graph: graph
} );

// add some data every so often

var messages = [
	"Changed home page welcome message",
	"Minified JS and CSS",
	"Changed button color from blue to green",
	"Refactored SQL query to use indexed columns",
	"Added additional logging for debugging",
	"Fixed typo",
	"Rewrite conditional logic for clarity",
	"Added documentation for new methods"
];

/*setInterval( function() {
	random.removeData(seriesData);
	random.addData(seriesData);
	graph.update();

}, 3000 );

function addAnnotation(force) {
	if (messages.length > 0 && (force || Math.random() >= 0.95)) {
		annotator.add(seriesData[2][seriesData[2].length-1].x, messages.shift());
		annotator.update();
	}
}

addAnnotation(true);
setTimeout( function() { setInterval( addAnnotation, 6000 ) }, 6000 );

var previewXAxis = new Rickshaw.Graph.Axis.Time({
	graph: preview.previews[0],
	timeFixture: new Rickshaw.Fixtures.Time.Local(),
	ticksTreatment: ticksTreatment
});

previewXAxis.render();*/


}


</script>

</body>
\end{lstlisting}
