\lstset{language=bash}
\begin{lstlisting}[frame=single, basicstyle=\linespread{0.85}\ttfamily\tiny, caption=/etc/network/interfaces, label=etc_net]
# This file describes the network interfaces available
# on your system and how to activate them.
# For more information, see interfaces(5).

# The loopback network interface
auto lo
iface lo inet loopback

# The primary network interface
#auto eth0
#iface eth0 inet dhcp
# Example to keep MAC address between reboots
#hwaddress ether DE:AD:BE:EF:CA:FE

# The secondary network interface
#auto eth1
#iface eth1 inet dhcp

# WiFi Example
auto wlan0
allow-hotplug wlan0
iface wlan0 inet static
#allow-hotplug wlan0
#auto lo
#iface lo inet loopback
#auto eth0
#iface eth0 inet static
address 192.168.1.155
netmask 255.255.252.0
network 192.168.1.0
gateway 192.168.1.1
#pre-up ifconfig eth0 hw ether 00:01:02:03:05:14
dns-nameservers  143.107.225.6 143.107.182.2 8.8.8.8
wpa-ssid        "nome_da_rede"
wpa-psk         "senha1"
#iface wlan0 inet dhcp
#    wpa-ssid "outra_rede"
#    wpa-psk  "senha2"

# Ethernet/RNDIS gadget (g_ether)
# ... or on host side, usbnet and random hwaddr
# Note on some boards, usb0 is automaticly 
#setup with an init script
iface usb0 inet static
address 192.168.7.2
netmask 255.255.255.0
network 192.168.7.0
gateway 192.168.7.1
\end{lstlisting}
