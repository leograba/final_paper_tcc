\lstset{language=HTML}
\begin{lstlisting}[frame=single, basicstyle=\linespread{0.85}\ttfamily\tiny, caption=P�gina \textit{sobre} da interface web, label=about_php]
<!DOCTYPE html>
<html>
<head>
    <meta charset="UTF-8">
    <title>About</title>
    <link href='https://fonts.googleapis.com/css?family=Roboto&subset=latin,latin-ext' rel='stylesheet' type='text/css'>
    <link rel="stylesheet" type="text/css" href="./css/config.css">
    <link rel="stylesheet" type="text/css" href="./css/buttons.css">
    <link rel="icon" type="image/png" href="./img/beer2.png">
    <script src="https://ajax.googleapis.com/ajax/libs/jquery/1.11.3/jquery.min.js"></script>
    <script type="text/javascript" src="./lib/header.js"></script>
	<script> 
		$(function(){headerPHP("./lib/header.php");});//add header 
	</script>
</head>

<body style="display:none;">
    <?php // include './lib/header.php'; echo generateHeader();?>
    <h1>Cervejaria do Futuro</h1>
    <p>Trabalho de Conclus�o de Curso de Engenharia El�trica - �nfase em Eletr�nica</p>
    <p>Escola de Engenharia de S�o Carlos - Universidade de S�o Paulo (EESC-USP)</p>
    <p>Professor orientador: Evandro Lu�s Linhari Rodrigues</p>
    <p>Aluno: Leonardo Graboski Veiga</p>
    <ul>
        <li>Email: leogveiga@gmail.com</li>
        <li>Github: <a style="font-size:1em;" href="https://github.com/leograba/final_paper_tcc">https://github.com/leograba/final_paper_tcc</a></li>
    </ul>
    
    <br>

    <p>Objetivo do Projeto: Desenvolver um sistema inovador de produ��o de cerveja
    , f�cil de usar, divertido, confi�vel e dispon�vel em qualquer lugar</p>
    <br>
    
    <p style="font-size:0.75em;">Esta interface foi desenvolvida para ser visualizada em desktop, com o navegador Google Chrome.
    Outros navegadores e/ou dispositivos m�veis podem apresentar problemas de design ou funcionalidade</p>
</body>
</html>
\end{lstlisting}
