\lstset{language=C}
\begin{lstlisting}[frame=single, basicstyle=\linespread{0.85}\ttfamily, caption=C�digo-fonte de caracteriza��o do funcionamento do servo-motor com AT89S52, label=at_servo_code]
#include <at89s8252.h>
/*********DEFINI��O DOS PORTS E CONSTANTES**************/
#define	RS		P1_0
#define	RW		P1_1
#define	E		P1_2
#define	LCD		P1
#define	LINE0		0x80
#define	LINE1		0xC0
#define 	SERVO		P2_0
/********************************************************/

/*************** VARI�VEIS EXTERNAS*********************/
volatile unsigned int estouro=0;
volatile unsigned char ciclo=25;
/********************************************************/

/**************fun��es de main.c*******************/
void config();
void tmr0() __interrupt 1;
void tmr1() __interrupt 3;
void ext_int0() __interrupt 0;
void ext_int1() __interrupt 2;
void inicia_lcd();
void envia_lcd(unsigned char);
void envia_str_lcd(unsigned char [],unsigned char);
void reset_lcd();
void atraso(unsigned char);
/****************fim das fun��es*******************/
void config(){
	TMOD=0x12;//timer 1 como timer(16b)) e timer 0 como timer(8b auto reload))
	IE=0x8A;//interrup��o geral e por timer 0 e 1 ativadas
	TL0=TH0=0xCE;//preset para 50us
	TH1=0xB1;//prepara para 20ms
	TL1=0xDF;
	EX0=EX1=1;	//habilita interrup��o externa
	TR1=1;//liga ambos os timers
	TR0=1;
	PT0=PT1=1;//prioridade de interrup��o maior para os timers
}

void tmr0() __interrupt 1{
	estouro++;
	if(estouro==ciclo){//se deu a largura de pulso
		estouro=0;//zera o controle
		SERVO=0;
		TR0=0;//desliga o timer
	}
}

void tmr1() __interrupt 3{
	TH1=0xB1;//prepara para 20ms
	TL1=0xDF;
	TL0=TH0=0xD8;//preset para 40us(12MHz) ou 20us(24MHz)
	TR0=1;//liga o timer 0
	SERVO=1;
}

void ext_int0() __interrupt 0{
	unsigned int atraso;
	//if(ciclo<126){//valor pr�tico para 24MHz -> 2,5ms
		ciclo++;
	//}
	for(atraso=0;atraso!=0xffff;atraso++);
}

void ext_int1() __interrupt 2{
	unsigned int atraso;
	//if(ciclo>24){//valor pr�tico para 24MHz -> 0,5ms
		ciclo--;
	//}
	for(atraso=0;atraso!=0xffff;atraso++);
}

void inicia_lcd(){
	const unsigned char instr[]={0x06,0x0C,0x28,0x01};
	unsigned char colorado,invicta;
	RS=RW=E=0;
	for(colorado=0;colorado!=60;colorado++){//gera um atraso de 15ms de inicializa��o
		atraso(250);
	}
	E=1;
	LCD=0x24;			//envia comando para opera��o em 4 bits
	E=0;
	for(invicta=0;invicta!=20;invicta++){		//gera um atraso de 5ms de inicializa��o
		atraso(250);
	}
	for(colorado=0;colorado!=4;colorado++){	//envia as instru��es de inicializa��o
		envia_lcd(instr[colorado]);
		for(invicta=0;invicta!=20;invicta++){	//gera um atraso de 5ms de inicializa��o
			atraso(250);
		}
	}
}

void envia_lcd(unsigned char letra){
	E=1;
	LCD=(0x0f&LCD)|(letra&0xf0);//m�scara para os bits de controle
	E=0;
	letra<<=4;//faz o swap dos nibbles
	E=1;
	LCD=(0x0f&LCD)|(letra&0xf0);//m�scara para os bits de controle
	E=0;
	atraso(250);//espera enviar o dado
}

void envia_str_lcd(unsigned char string[],unsigned char pos){
	unsigned char way;
	RS=0;
	envia_lcd(pos);
	RS=1;
	for(way=0;string[way]!='\0';way++){
		envia_lcd(string[way]);
	}
}

void reset_lcd(){
	unsigned char coruja;
	RS=RW=0;
	E=1;
	LCD=0x04;
	E=0;
	E=1;
	LCD=0x14;
	E=0;
	for(coruja=0;coruja!=8;coruja++){	//gera um atraso de 2ms de inicializa��o
		atraso(250);
	}
}

void atraso(unsigned char tempo){
	//essa rotina n�o � recomendada para tempos de menos que 30us(22us na verdade, massss....)
	//as instru��es a seguir corrigem o offset da fun��o
	tempo-=15;	//1us
	tempo/=6;	//8us
	//cada la�o while demora 6us
	while(tempo){	//enquanto o tempo em us n�o for zero
		--tempo;//decrementa a cada ciclo do la�o while
	}
}

void main(){
	unsigned char temp;
	inicia_lcd();
	envia_str_lcd("Ciclo:",0x80);
	config();
	while(1){
		RS=0;
		envia_lcd(0x86);
		RS=1;
		temp=ciclo/100;
		envia_lcd(temp+'0');
		temp=ciclo-temp*100;
		envia_lcd((temp/10)+'0');
		temp-=(temp/10)*10;
		envia_lcd(temp+'0');
	}
}
\end{lstlisting}
